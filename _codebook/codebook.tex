\documentclass[a4paper,10pt,twocolumn,oneside]{article}
\setlength{\columnsep}{10pt}                                                                    %兩欄模式的間距
\setlength{\columnseprule}{0pt}                                                                %兩欄模式間格線粗細

\usepackage{amsthm}								%定義,例題
\usepackage{amssymb}
%\usepackage[margin=2cm]{geometry}
\usepackage{fontspec}								%設定字體
\usepackage{color}
\usepackage[x11names]{xcolor}
\usepackage{listings}								%顯示code用的
%\usepackage[Glenn]{fncychap}						%排版,頁面模板
\usepackage{fancyhdr}								%設定頁首頁尾
\usepackage{graphicx}								%Graphic
\usepackage{enumerate}
\usepackage{titlesec}
\usepackage{amsmath}
\usepackage[CheckSingle, CJKmath]{xeCJK}
% \usepackage{CJKulem}

%\usepackage[T1]{fontenc}
\usepackage{amsmath, courier, listings, fancyhdr, graphicx}
\topmargin=0pt
\headsep=5pt
\textheight=780pt
\footskip=0pt
\voffset=-40pt
\textwidth=545pt
\marginparsep=0pt
\marginparwidth=0pt
\marginparpush=0pt
\oddsidemargin=0pt
\evensidemargin=0pt
\hoffset=-42pt

\titlespacing\subsection{0pt}{5pt plus 4pt minus 2pt}{0pt plus 2pt minus 2pt}


%\renewcommand\listfigurename{圖目錄}
%\renewcommand\listtablename{表目錄} 

%%%%%%%%%%%%%%%%%%%%%%%%%%%%%

\setmainfont{Consolas}				%主要字型
%\setmonofont{Monaco}				%主要字型
\setmonofont{Consolas}
\setCJKmainfont{Noto Sans CJK TC}
% \setCJKmainfont{Consolas}			%中文字型
%\setmainfont{sourcecodepro}
\XeTeXlinebreaklocale "zh"						%中文自動換行
\XeTeXlinebreakskip = 0pt plus 1pt				%設定段落之間的距離
\setcounter{secnumdepth}{3}						%目錄顯示第三層

%%%%%%%%%%%%%%%%%%%%%%%%%%%%%
\makeatletter
\lst@CCPutMacro\lst@ProcessOther {"2D}{\lst@ttfamily{-{}}{-{}}}
\@empty\z@\@empty
\makeatother
\lstset{											% Code顯示
language=C++,										% the language of the code
basicstyle=\footnotesize\ttfamily, 						% the size of the fonts that are used for the code
%numbers=left,										% where to put the line-numbers
numberstyle=\footnotesize,						% the size of the fonts that are used for the line-numbers
stepnumber=1,										% the step between two line-numbers. If it's 1, each line  will be numbered
numbersep=5pt,										% how far the line-numbers are from the code
backgroundcolor=\color{white},					% choose the background color. You must add \usepackage{color}
showspaces=false,									% show spaces adding particular underscores
showstringspaces=false,							% underline spaces within strings
showtabs=false,									% show tabs within strings adding particular underscores
frame=false,											% adds a frame around the code
tabsize=2,											% sets default tabsize to 2 spaces
captionpos=b,										% sets the caption-position to bottom
breaklines=true,									% sets automatic line breaking
breakatwhitespace=false,							% sets if automatic breaks should only happen at whitespace
escapeinside={\%*}{*)},							% if you want to add a comment within your code
morekeywords={constexpr},									% if you want to add more keywords to the set
keywordstyle=\bfseries\color{Blue1},
commentstyle=\itshape\color{Red4},
stringstyle=\itshape\color{Green4},
}

%%%%%%%%%%%%%%%%%%%%%%%%%%%%%

\begin{document}
\pagestyle{fancy}
\fancyfoot{}
%\fancyfoot[R]{\includegraphics[width=20pt]{ironwood.jpg}}
\fancyhead[L]{--}
\fancyhead[R]{\thepage}
\renewcommand{\headrulewidth}{0.4pt}
\renewcommand{\contentsname}{Contents} 

\scriptsize
\tableofcontents
%%%%%%%%%%%%%%%%%%%%%%%%%%%%%

%\newpage

\section{Basic}
\subsection{Shell script}
\lstinputlisting{1_Basic/Shell_script.cpp}
\subsection{readchar}
\lstinputlisting{1_Basic/readchar.cpp}
\subsection{FastIO}
\lstinputlisting{1_Basic/Fast_IO.cpp}
\subsection{64bit-multiplication}
\lstinputlisting{1_Basic/64bit_multiplication.cpp}
\subsection{fast calculate for RoHash}
\lstinputlisting{1_Basic/fast_count_prime_for_rolling_hash.cpp}
\subsection{mt19937}
\lstinputlisting{1_Basic/mt19937.cpp}
\subsection{Black Magic}
\lstinputlisting{1_Basic/black_magic.cpp}
% \subsection{Texas hold'em}
% \lstinputlisting{1_Basic/Texas_holdem.cpp}


\section{Graph}
\subsection{BCC Vertex*} % test by CF 102512 A
\lstinputlisting{2_Graph/BCC_Vertex.cpp}
\subsection{Bridge*} % test by TIOJ 1879
\lstinputlisting{2_Graph/Bridge.cpp}
\subsection{Delete Bridge} 
\lstinputlisting{2_Graph/Delete_bridge.cpp}
\subsection{2SAT (SCC)*} % SCC test by TIOJ 2143, 2SAT test by ARC069 F
\lstinputlisting{2_Graph/2SAT.cpp}
\subsection{Johnson Algorithm} % SCC test by TIOJ 2143, 2SAT test by ARC069 F
\lstinputlisting{2_Graph/Johnson_Algorithm.cpp}
\subsection{MinimumMeanCycle*} % test by TIOJ 1934
\lstinputlisting{2_Graph/MinimumMeanCycle.cpp}
\subsection{Virtual Tree*} % test by luogu P2495
\lstinputlisting{2_Graph/Virtual_Tree.cpp}
\subsection{Minimum Steiner Tree*} % test by luogu P6192
\lstinputlisting{2_Graph/MinimumSteinerTree.cpp}
\subsection{Dominator Tree*} % test by CF 100513 L
\lstinputlisting{2_Graph/Dominator_Tree.cpp}
\subsection{Vizing's theorem}
\lstinputlisting{2_Graph/Vizing.cpp}
\subsection{Erdos–Gallai theorem}
\lstinputlisting{2_Graph/Erdos-Gallai.cpp}
% \subsection{Theory}
% \begin{footnotesize}
% $|$Maximum independent edge set$|=|V|-|$Minimum edge cover$|$\\
% $|$Maximum independent set$|=|V|-|$Minimum vertex cover$|$\\
% \end{footnotesize}


\section{Data Structure}
\subsection{Leftist Tree}
\lstinputlisting{3_Data_Structure/Leftist_Tree.cpp}
\subsection{Heavy light Decomposition}
\lstinputlisting{3_Data_Structure/Heavy_light_Decomposition.cpp}
\subsection{Centroid Decomposition*} % test by TIOJ 1171
\lstinputlisting{3_Data_Structure/Centroid_Decomposition.cpp}
% \subsection{Smart Pointer}
% \lstinputlisting{3_Data_Structure/Smart_Pointer.cpp}
\subsection{LiChaoST}
\lstinputlisting{3_Data_Structure/LiChaoST.cpp}
\subsection{Link cut tree*} % test by luogu P3690
\lstinputlisting{3_Data_Structure/link_cut_tree.cpp}
\subsection{KDTree}
\lstinputlisting{3_Data_Structure/KDTree.cpp}
\subsection{DSU Undo}
\lstinputlisting{3_Data_Structure/dsu_undo.cpp}
% \subsection{Range Chmin Chmax Add Range Sum*} % test by Lib-Checker Range Chmin Chmax Add Range Sum
% \lstinputlisting{3_Data_Structure/Range_Chmin_Chmax_Add_Range_Sum.cpp}


\section{Flow/Matching}
\subsection{Dinic}
\lstinputlisting{4_Flow_Matching/Dinic_.cpp}
\subsection{Bipartite Matching} % test by Lib-Checker Matching on Bipartite Graph
\lstinputlisting{4_Flow_Matching/Bipartite_Matching.cpp}
\subsection{Kuhn Munkres}
\lstinputlisting{4_Flow_Matching/Kuhn_Munkres.cpp}
\subsection{MincostMaxflow}
\lstinputlisting{4_Flow_Matching/MCMF.cpp}
\subsection{Maximum Simple Graph Matching*} % test by TIOJ 1504
\lstinputlisting{4_Flow_Matching/Maximum_Simple_Graph_Matching.cpp}
\subsection{Minimum Weight Matching (Clique version)*} % test by uoj.ac 81
\lstinputlisting{4_Flow_Matching/Minimum_Weight_Matching.cpp}
\subsection{SW-mincut}
\lstinputlisting{4_Flow_Matching/SW-mincut.cpp}
% \subsection{BoundedFlow(Dinic*)} % test by NTUJ 184(Only Dinic)
% \lstinputlisting{4_Flow_Matching/BoundedFlow.cpp}
\subsection{Gomory Hu tree*} % test by BZOJ 4519
\lstinputlisting{4_Flow_Matching/Gomory_Hu_tree.cpp}
\subsection{Minimum Cost Circulation}
\lstinputlisting{4_Flow_Matching/MinCostCirculation.cpp}
\subsection{Flow Models}
% \normalsize
\begin{itemize}
    \itemsep-0.3em
    \item Maximum/Minimum flow with lower bound / Circulation problem
    \vspace{-1em}
    \begin{enumerate}
        \itemsep-0.3em
        \item Construct super source $S$ and sink $T$.
        \item For each edge $(x, y, l, u)$, connect $x \rightarrow y$ with capacity $u - l$.
        \item For each vertex $v$, denote by $in(v)$ the difference between the sum of incoming lower bounds and the sum of outgoing lower bounds.
        \item If $in(v) > 0$, connect $S \rightarrow v$ with capacity $in(v)$, otherwise, connect $v \rightarrow T$ with capacity $-in(v)$.
        \begin{itemize}
            \itemsep-0.2em
            \item To maximize, connect $t \rightarrow s$ with capacity $\infty$ (skip this in circulation problem), and let $f$ be the maximum flow from $S$ to $T$. If $f \neq \sum_{v \in V, in(v) > 0}{in(v)}$, there's no solution. Otherwise, the maximum flow from $s$ to $t$ is the answer.
            \item To minimize, let $f$ be the maximum flow from $S$ to $T$. Connect $t \rightarrow s$ with capacity $\infty$ and let the flow from $S$ to $T$ be $f^\prime$. If $f + f^\prime \neq \sum_{v \in V, in(v) > 0}{in(v)}$, there's no solution. Otherwise, $f^\prime$ is the answer.
        \end{itemize}
        \item The solution of each edge $e$ is $l_e + f_e$, where $f_e$ corresponds to the flow of edge $e$ on the graph.
    \end{enumerate}
    \item Construct minimum vertex cover from maximum matching $M$ on bipartite graph $(X, Y)$
    \vspace{-1em}
    \begin{enumerate}
        \itemsep-0.3em
        \item Redirect every edge: $y \rightarrow x$ if $(x, y) \in M$, $x \rightarrow y$ otherwise.
        \item DFS from unmatched vertices in $X$.
        \item $x \in X$ is chosen iff $x$ is unvisited.
        \item $y \in Y$ is chosen iff $y$ is visited.
    \end{enumerate}
    \item Minimum cost cyclic flow
    \vspace{-0.5em}
    \begin{enumerate}
        \itemsep-0.3em
        \item Consruct super source $S$ and sink $T$
        \item For each edge $(x, y, c)$, connect $x \rightarrow y$ with $(cost, cap) = (c, 1)$ if $c > 0$, otherwise connect $y \rightarrow x$ with $(cost, cap) = (-c, 1)$
        \item For each edge with $c < 0$, sum these cost as $K$, then increase $d(y)$ by 1, decrease $d(x)$ by 1
        \item For each vertex $v$ with $d(v) > 0$, connect $S \rightarrow v$ with $(cost, cap) = (0, d(v))$
        \item For each vertex $v$ with $d(v) < 0$, connect $v \rightarrow T$ with $(cost, cap) = (0, -d(v))$
        \item Flow from $S$ to $T$, the answer is the cost of the flow $C + K$
    \end{enumerate}
    \item Maximum density induced subgraph
    \vspace{-1em}
    \begin{enumerate}
        \itemsep-0.3em
        \item Binary search on answer, suppose we're checking answer $T$
        \item Construct a max flow model, let $K$ be the sum of all weights
        \item Connect source $s \rightarrow v$, $v \in G$ with capacity $K$
        \item For each edge $(u, v, w)$ in $G$, connect $u \rightarrow v$ and $v \rightarrow u$ with capacity $w$
        \item For $v \in G$, connect it with sink $v \rightarrow t$ with capacity $K + 2T - (\sum_{e \in E(v)}{w(e)}) - 2w(v)$
        \item $T$ is a valid answer if the maximum flow $f < K \lvert V \rvert$
    \end{enumerate}
    \item Minimum weight edge cover
    \vspace{-1em}
    \begin{enumerate}
        \itemsep-0.3em
      \item For each $v \in V$ create a copy $v^\prime$, and connect $u^\prime \to v^\prime$ with weight $w(u, v)$.
      \item Connect $v \to v^\prime$ with weight $2\mu(v)$, where $\mu(v)$ is the cost of the cheapest edge incident to $v$.
      \item Find the minimum weight perfect matching on $G^\prime$.
    \end{enumerate}
    \item Project selection problem
    \vspace{-1em}
    \begin{enumerate}
      \itemsep-0.3em
      \item If $p_v > 0$, create edge $(s, v)$ with capacity $p_v$; otherwise, create edge $(v, t)$ with capacity $-p_v$.
      \item Create edge $(u, v)$ with capacity $w$ with $w$ being the cost of choosing $u$ without choosing $v$.
      \item The mincut is equivalent to the maximum profit of a subset of projects.
    \end{enumerate}
    \item 0/1 quadratic programming
    \vspace{-1em}
    \[ \sum_x{c_xx} + \sum_y{c_y\bar{y}} + \sum_{xy}c_{xy}x\bar{y} + \sum_{xyx^\prime y^\prime}c_{xyx^\prime y^\prime}(x\bar{y} + x^\prime\bar{y^\prime}) \]
    can be minimized by the mincut of the following graph:
    \begin{enumerate}
      \itemsep-0.3em
      \item Create edge $(x, t)$ with capacity $c_x$ and create edge $(s, y)$ with capacity $c_y$.
      \item Create edge $(x, y)$ with capacity $c_{xy}$.
      \item Create edge $(x, y)$ and edge $(x^\prime, y^\prime)$ with capacity $c_{xyx^\prime y^\prime}$.
    \end{enumerate}
\end{itemize}

% \subsection{isap}
% \lstinputlisting{4_Flow_Matching/isap.cpp}


\section{String}
\subsection{KMP}
\lstinputlisting{5_String/KMP.cpp}
\subsection{Another KMP}
\lstinputlisting{5_String/KMP_.cpp}
\subsection{Z-value*} % test by Lib-Checker Z Algorithm
\lstinputlisting{5_String/Z-value.cpp}
\subsection{Manacher*} % test by TIOJ 1276
\lstinputlisting{5_String/Manacher.cpp}
\subsection{Suffix Array+LCP}
\lstinputlisting{5_String/Suffix_Array.cpp}
\subsection{SAIS*} % test by TIOJ 1927
\lstinputlisting{5_String/SAIS.cpp}
\subsection{Aho-Corasick Automatan}
\lstinputlisting{5_String/Aho-Corasick_Automatan.cpp}
\subsection{Smallest Rotation}
\lstinputlisting{5_String/Smallest_Rotation.cpp}
\subsection{De Bruijn sequence*} % test by CF 102001 C
\lstinputlisting{5_String/De_Bruijn_sequence.cpp}
\subsection{SAM}
\lstinputlisting{5_String/SAM.cpp}
\subsection{PalTree*} % test by APIO 2014 palindrome
\lstinputlisting{5_String/PalTree.cpp}


\section{Math}
\subsection{ax+by=gcd*} % test by NTUJ 110
\lstinputlisting{6_Math/ax+by=gcd.cpp}
\subsection{floor and ceil}
\lstinputlisting{6_Math/floor_ceil.cpp}
\subsection{Gaussian integer gcd}
\lstinputlisting{6_Math/Gaussian_gcd.cpp}
\subsection{Joseph problem}
\lstinputlisting{6_Math/joseph_problem.cpp}
% \subsection{floor sum*} % test by AtCoder Library Practice Contest C, CF 100920 J
% \lstinputlisting{6_Math/floor_sum.cpp}
\subsection{Miller Rabin*} % test by NTUJ 1237
\lstinputlisting{6_Math/Miller_Rabin.cpp}
% \subsection{Big number}
% \lstinputlisting{6_Math/Big_number.cpp}
\subsection{Fraction}
\lstinputlisting{6_Math/Fraction.cpp}
\subsection{Simultaneous Equations}
\lstinputlisting{6_Math/Simultaneous_Equations.cpp}
\subsection{Pollard Rho*} % test by Lib-Checker Factorize
\lstinputlisting{6_Math/Pollard_Rho.cpp}
\subsection{Simplex Algorithm}
\lstinputlisting{6_Math/Simplex_Algorithm.cpp}
\subsubsection{Construction}
% \normalsize
Standard form: maximize $\mathbf{c}^T\mathbf{x}$ subject to $A\mathbf{x} \leq \mathbf{b}$ and $\mathbf{x} \geq 0$. \\
Dual LP: minimize $\mathbf{b}^T\mathbf{y}$ subject to $A^T\mathbf{y} \geq \mathbf{c}$ and $\mathbf{y} \geq 0$. \\
$\bar{\mathbf{x}}$ and $\bar{\mathbf{y}}$ are optimal if and only if for all $i \in [1, n]$, either $\bar{x}_i = 0$ or $\sum_{j=1}^{m}A_{ji}\bar{y}_j = c_i$ holds and for all $i \in [1, m]$ either $\bar{y}_i = 0$ or $\sum_{j=1}^{n}A_{ij}\bar{x}_j = b_j$ holds.

\begin{enumerate}
    \itemsep-0.5em
    \item In case of minimization, let $c^\prime_i = -c_i$
    \item $\sum_{1 \leq i \leq n}{A_{ji}x_i} \geq b_j \rightarrow \sum_{1 \leq i \leq n}{-A_{ji}x_i} \leq -b_j$
    \item $\sum_{1 \leq i \leq n}{A_{ji}x_i} = b_j$ 
        \vspace{-0.5em}
        \begin{itemize}
            \itemsep-0.5em
            \item $\sum_{1 \leq i \leq n}{A_{ji}x_i} \leq b_j$
            \item $\sum_{1 \leq i \leq n}{A_{ji}x_i} \geq b_j$
        \end{itemize}
    \item If $x_i$ has no lower bound, replace $x_i$ with $x_i - x_i^\prime$
\end{enumerate}

\subsection{Schreier-Sims Algorithm*} % test by XVI Opencup GP of Ekaterinburg H
\lstinputlisting{6_Math/SchreierSims.cpp}
\subsection{chineseRemainder}
\lstinputlisting{6_Math/chineseRemainder.cpp}
\subsection{Factorial without prime factor*} % test by luogu P4720
\lstinputlisting{6_Math/fac_no_p.cpp}
\subsection{QuadraticResidue*} % test by Lib-Checker Sqrt Mod
\lstinputlisting{6_Math/QuadraticResidue.cpp}
\subsection{PiCount*} % test by luogu P7884
\lstinputlisting{6_Math/PiCount.cpp}
\subsection{Discrete Log*} % test by Lib-Checker Discrete Logarithm
\lstinputlisting{6_Math/DiscreteLog.cpp}
\subsection{Primes}
\lstinputlisting{6_Math/Primes.cpp}
\subsection{Theorem}
\begin{itemize}
\item Kirchhoff's Theorem

Denote $L$ be a $n \times n$ matrix as the Laplacian matrix of graph $G$, where $L_{ii} = d(i)$, $L_{ij} = -c$ where $c$ is the number of edge $(i, j)$ in $G$.
\begin{itemize}
    \itemsep-0.5em
    \item The number of undirected spanning in $G$ is $\lvert \det(\tilde{L}_{11}) \rvert$.
    \item The number of directed spanning tree rooted at $r$ in $G$ is $\lvert \det(\tilde{L}_{rr}) \rvert$.
\end{itemize}

\item Tutte's Matrix

Let $D$ be a $n \times n$ matrix, where $d_{ij} = x_{ij}$ ($x_{ij}$ is chosen uniformly at random) if $i < j$ and $(i, j) \in E$, otherwise $d_{ij} = -d_{ji}$. $\frac{rank(D)}{2}$ is the maximum matching on $G$.

\item Cayley's Formula

\begin{itemize}
    \itemsep-0.5em
  \item Given a degree sequence $d_1, d_2, \ldots, d_n$ for each \textit{labeled} vertices, there are $\frac{(n - 2)!}{(d_1 - 1)!(d_2 - 1)!\cdots(d_n - 1)!}$ spanning trees.
  \item Let $T_{n, k}$ be the number of \textit{labeled} forests on $n$ vertices with $k$ components, such that vertex $1, 2, \ldots, k$ belong to different components. Then $T_{n, k} = kn^{n - k - 1}$.
\end{itemize}

\item Erdős–Gallai theorem 

A sequence of nonnegative integers $d_1\ge\cdots\ge d_n$ can be represented as the degree sequence of a finite simple graph on $n$ vertices if and only if $d_1+\cdots+d_n$ is even and $\displaystyle\sum_{i-1}^kd_i\le k(k-1)+\displaystyle\sum_{i=k+1}^n\min(d_i,k)$ holds for every $1\le k\le n$.

\item Gale–Ryser theorem

A pair of sequences of nonnegative integers $a_1\ge\cdots\ge a_n$ and $b_1,\ldots,b_n$ is bigraphic if and only if $\displaystyle\sum_{i=1}^n a_i=\displaystyle\sum_{i=1}^n b_i$ and $\displaystyle\sum_{i=1}^k a_i\le \displaystyle\sum_{i=1}^n\min(b_i,k)$ holds for every $1\le k\le n$.

\item Fulkerson–Chen–Anstee theorem

A sequence $(a_1,b_1),\ldots,(a_n,b_n)$ of nonnegative integer pairs with $a_1\ge\cdots\ge a_n$ is digraphic if and only if $\displaystyle\sum_{i=1}^n a_i=\displaystyle\sum_{i=1}^n b_i$ and $\displaystyle\sum_{i=1}^k a_i\le \displaystyle\sum_{i=1}^k\min(b_i,k-1)+\displaystyle\sum_{i=k+1}^n\min(b_i,k)$ holds for every $1\le k\le n$.

\item Möbius inversion formula

\begin{itemize}
    \itemsep-0.5em
  \item $f(n)=\sum_{d\mid n}g(d)\Leftrightarrow g(n)=\sum_{d\mid n}\mu(d)f(\frac{n}{d})$
  \item $f(n)=\sum_{n\mid d}g(d)\Leftrightarrow g(n)=\sum_{n\mid d}\mu(\frac{d}{n})f(d)$
\end{itemize}

\item Spherical cap

\begin{itemize}
    \itemsep-0.5em
  \item A portion of a sphere cut off by a plane.
  \item $r$: sphere radius, $a$: radius of the base of the cap, $h$: height of the cap, $\theta$: $\arcsin(a/r)$.
  \item Volume $=\pi h^2(3r-h)/3=\pi h(3a^2+h^2)/6=\pi r^3(2+\cos\theta)(1-\cos\theta)^2/3$.
  \item Area $=2\pi rh=\pi(a^2+h^2)=2\pi r^2(1-\cos\theta)$.
\end{itemize}

\end{itemize}

\subsection{Euclidean Algorithms}
\begin{itemize}
  \itemsep-0.5em
  \item $m = \lfloor\frac{an + b}{c}\rfloor$
  \item Time complexity: $O(\log{n})$
\end{itemize}

$$ \begin{aligned}
  f(a, b, c, n) &= \sum_{i = 0}^{n}\lfloor\frac{ai + b}{c}\rfloor \\
  &= \begin{cases} 
    \lfloor\frac{a}{c}\rfloor \cdot \frac{n(n + 1)}{2} + \lfloor\frac{b}{c}\rfloor \cdot (n + 1) \\ + f(a\text{ mod } c, b\text{ mod } c, c, n), & a \geq c \lor b \geq c \\ 
    0, & n < 0 \lor a = 0 \\
    nm - f(c, c - b - 1, a, m - 1), & \text{otherwise} 
  \end{cases} 
\end{aligned} $$
$$ \begin{aligned}
  g(a, b, c, n) &= \sum_{i = 0}^{n}i\lfloor\frac{ai + b}{c}\rfloor \\
  &= \begin{cases}
    \lfloor{\frac{a}{c}}\rfloor \cdot \frac{n(n + 1)(2n + 1)}{6} + \lfloor\frac{b}{c}\rfloor \cdot \frac{n(n + 1)}{2} \\ + g(a\text{ mod } c, b\text{ mod } c, c, n), & a \geq c \lor b \geq c \\
    0, & n < 0 \lor a = 0 \\
    \frac{1}{2} \cdot (n(n + 1)m - f(c, c - b - 1, a, m - 1) \\ - h(c, c - b - 1, a, m - 1)), & \text{otherwise}
  \end{cases}
\end{aligned} $$
$$ \begin{aligned}
  h(a, b, c, n) &= \sum_{i = 0}^{n}\lfloor\frac{ai + b}{c}\rfloor^2 \\
  &= \begin{cases}
    \lfloor\frac{a}{c}\rfloor^2 \cdot \frac{n(n + 1)(2n + 1)}{6} + \lfloor\frac{b}{c}\rfloor^2 \cdot (n + 1) \\ + \lfloor\frac{a}{c}\rfloor \cdot \lfloor\frac{b}{c}\rfloor \cdot n(n + 1) \\ + h(a\text{ mod } c, b\text{ mod } c, c, n) \\ + 2\lfloor\frac{a}{c}\rfloor \cdot g(a\text{ mod } c, b\text{ mod } c, c, n) \\ + 2\lfloor\frac{b}{c}\rfloor \cdot f(a\text{ mod } c, b\text{ mod } c, c, n), & a \geq c \lor b \geq c \\
    0, & n < 0 \lor a = 0 \\
    nm(m + 1) - 2g(c, c - b - 1, a, m - 1) \\ - 2f(c, c - b - 1, a, m - 1) - f(a, b, c, n), & \text{otherwise}
  \end{cases}
\end{aligned} $$


\subsection{General Purpose Numbers}
\begin{itemize}
\item Bernoulli numbers

$B_0-1,B_1^{\pm}=\pm\frac{1}{2},B_2=\frac{1}{6},B_3=0$

$\displaystyle\sum_{j=0}^m\binom{m+1}{j}B_j=0$, EGF is $B(x) = \frac{x}{e^x - 1}=\displaystyle\sum_{n=0}^\infty B_n\frac{x^n}{n!}$.

$S_m(n)=\displaystyle\sum_{k=1}^nk^m=\frac{1}{m+1}\sum_{k=0}^m\binom{m+1}{k}B^{+}_kn^{m+1-k}$

\item Stirling numbers of the second kind
Partitions of $n$ distinct elements into exactly $k$ groups. 

$S(n, k) = S(n - 1, k - 1) + kS(n - 1, k), S(n, 1) = S(n, n) = 1$

$S(n, k) = \frac{1}{k!}\sum_{i=0}^{k}(-1)^{k-i}{k \choose i}i^n$

$x^n     = \sum_{i=0}^{n} S(n, i) (x)_i$

\item Pentagonal number theorem

$\displaystyle\prod_{n=1}^{\infty}(1-x^n)=1+\sum_{k=1}^{\infty}(-1)^k\left(x^{k(3k+1)/2} + x^{k(3k-1)/2}\right)$

\item Catalan numbers

$C^{(k)}_n = \displaystyle \frac{1}{(k - 1)n + 1}\binom{kn}{n}$

$C^{(k)}(x) = 1 + x [C^{(k)}(x)]^k$
\end{itemize}

\subsection{Tips for Generating Functions}
\begin{itemize}
\item Ordinary Generating Function
$A(x) = \sum_{i\ge 0} a_ix^i$
\begin{itemize}
    \itemsep-0.5em
    \item $A(rx)             \Rightarrow r^na_n$
    \item $A(x) + B(x)       \Rightarrow a_n + b_n$
    \item $A(x)B(x)          \Rightarrow \sum_{i=0}^{n} a_ib_{n-i}$
    \item $A(x)^k            \Rightarrow \sum_{i_1+i_2+\cdots+i_k=n} a_{i_1}a_{i_2}\ldots a_{i_k}$
    \item $xA(x)'            \Rightarrow na_n$
    \item $\frac{A(x)}{1-x}  \Rightarrow \sum_{i=0}^{n} a_i$
\end{itemize}
\item Exponential Generating Function
$A(x) = \sum_{i\ge 0} \frac{a_i}{i!}x_i$
\begin{itemize}
    \itemsep-0.5em
    \item $A(x) + B(x)       \Rightarrow a_n + b_n$
    \item $A^{(k)}(x)        \Rightarrow a_{n+k}$
    \item $A(x)B(x)          \Rightarrow \sum_{i=0}^{n} \binom{n}{i}a_ib_{n-i}$
    \item $A(x)^k            \Rightarrow \sum_{i_1+i_2+\cdots+i_k=n} \binom{n}{i_1, i_2, \ldots, i_k}a_{i_1}a_{i_2}\ldots a_{i_k}$
    \item $xA(x)             \Rightarrow na_n$
\end{itemize}
\item Special Generating Function
\begin{itemize}
    \itemsep-0.5em
    \item $(1+x)^n           = \sum_{i\ge 0} \binom{n}{i}x^i$
    \item $\frac{1}{(1-x)^n} = \sum_{i\ge 0} \binom{i}{n-1}x^i$
\end{itemize}
\end{itemize}


\section{Polynomial}
\subsection{Fast Fourier Transform}
\lstinputlisting{7_Polynomial/Fast_Fourier_Transform.cpp}
\subsection{Number Theory Transform*} % test by Lib-Checker Convolution
\lstinputlisting{7_Polynomial/Number_Theory_Transform.cpp}
\subsection{Fast Walsh Transform*} % test by luogu P6097
\lstinputlisting{7_Polynomial/Fast_Walsh_Transform.cpp}
\subsection{Polynomial Operation}
\lstinputlisting{7_Polynomial/Polynomial_Operation.cpp}
\subsection{Value Polynomial}
\lstinputlisting{7_Polynomial/Value_Poly.cpp}
\subsection{Newton's Method}
Given $F(x)$ where

$$ F(x) = \sum_{i=0}^{\infty}{\alpha_i(x - \beta)^i} $$

for $\beta$ being some constant. Polynomial $P$ such that $F(P) = 0$ can be found iteratively. Denote by $Q_k$ the polynomial such that $F(Q_k) = 0 \pmod {x^{2^k}}$, then

$$ Q_{k+1} = Q_k - \frac{F(Q_k)}{F^\prime(Q_k)} \pmod {x^{2^{k+1}}} $$


\section{Geometry}
\subsection{Default Code}
\lstinputlisting{8_Geometry/Default_code.cpp}
\subsection{Convex hull*} % test by Zerojudge b398
\lstinputlisting{8_Geometry/Convex_hull.cpp}
\subsection{External bisector}
\lstinputlisting{8_Geometry/external_bisector.cpp}
\subsection{Heart}
\lstinputlisting{8_Geometry/Heart.cpp}
\subsection{Minimum Enclosing Circle*} % test by TIOJ 1093
\lstinputlisting{8_Geometry/Minimum_Enclosing_Circle.cpp}
\subsection{Polar Angle Sort*} % test by NTUJ 2270
\lstinputlisting{8_Geometry/Polar_Angle_Sort.cpp}
\subsection{Intersection of two circles*} % test by TIOJ 1503
\lstinputlisting{8_Geometry/Intersection_of_two_circles.cpp}
\subsection{Intersection of polygon and circle*} % test by HDU 2892
\lstinputlisting{8_Geometry/Intersection_of_polygon_and_circle.cpp}
\subsection{Intersection of line and circle}
\lstinputlisting{8_Geometry/Intersection_of_line_and_circle.cpp}
\subsection{point in circle}
\lstinputlisting{8_Geometry/point_in_circle.cpp}
\subsection{Half plane intersection}
\lstinputlisting{8_Geometry/Half_plane_intersection.cpp}
\subsection{CircleCover*} % test by TIOJ 1503
\lstinputlisting{8_Geometry/CircleCover.cpp}
\subsection{3Dpoint*} % test by HDU 3662
\lstinputlisting{8_Geometry/3Dpoint.cpp}
\subsection{Convexhull3D*} % test by HDU 3662
\lstinputlisting{8_Geometry/Convexhull3D.cpp}
\subsection{DelaunayTriangulation*} % test by Zerojudge b370
\lstinputlisting{8_Geometry/DelaunayTriangulation.cpp}
\subsection{Triangulation Vonoroi*} % test by NPSC 2018 territory
\lstinputlisting{8_Geometry/Triangulation_Vonoroi.cpp}
\subsection{Tangent line of two circles}
\lstinputlisting{8_Geometry/Tangent_line_of_two_circles.cpp}
\subsection{minMaxEnclosingRectangle}
\lstinputlisting{8_Geometry/minMaxEnclosingRectangle.cpp}
\subsection{PointSegDist}
\lstinputlisting{8_Geometry/PointSegDist.cpp}
\subsection{PointInConvex}
\lstinputlisting{8_Geometry/PointInConvex.cpp}
\subsection{Minkowski Sum*} % test by Zerojudge b398
\lstinputlisting{8_Geometry/Minkowski_Sum.cpp}
\subsection{RotatingSweepLine}
\lstinputlisting{8_Geometry/rotatingSweepLine.cpp}

\section{Else}
\subsection{Mo's Alogrithm(With modification)}
\lstinputlisting{9_Else/Mos_Alogrithm_With_modification.cpp}
\subsection{Mo's Alogrithm On Tree}
\lstinputlisting{9_Else/Mos_Alogrithm_On_Tree.cpp}
\subsection{Hilbert Curve}
\lstinputlisting{9_Else/HilbertCurve.cpp}
\subsection{DynamicConvexTrick*} % test by TIOJ 1921
\lstinputlisting{9_Else/DynamicConvexTrick.cpp}
%\subsection{cyclicLCS}
%\lstinputlisting{9_Else/cyclicLCS.cpp}
\subsection{All LCS*} % test by Library Checker prefix-substring LCS
\lstinputlisting{9_Else/All_LCS.cpp}
% \subsection{DLX*} % test by TIOJ 1333, 1381
% \lstinputlisting{9_Else/DLX.cpp}
\subsection{Matroid Intersection}
Start from $S = \emptyset$. In each iteration, let 
\vspace{-0.5em}
\begin{itemize}
    \itemsep-0.5em
  \item $Y_1 = \{x \not\in S \mid S \cup \{x\} \in I_1 \}$
  \item $Y_2 = \{x \not\in S \mid S \cup \{x\} \in I_2 \}$
\end{itemize}
If there exists $x \in Y_1 \cap Y_2$, insert $x$ into $S$. Otherwise for each $x \in S, y \not\in S$, create edges
\vspace{-0.5em}
\begin{itemize}
    \itemsep-0.5em
  \item $x \to y$ if $S - \{x\} \cup \{y\} \in I_1$.
  \item $y \to x$ if $S - \{x\} \cup \{y\} \in I_2$.
\end{itemize}
Find a \textit{shortest} path (with BFS) starting from a vertex in $Y_1$ and ending at a vertex in $Y_2$ which doesn't pass through any other vertices in $Y_2$, and alternate the path. The size of $S$ will be incremented by 1 in each iteration. For the weighted case, assign weight $w(x)$ to vertex $x$ if $x \in S$ and $-w(x)$ if $x \not\in S$. Find the path with the minimum number of edges among all minimum length paths and alternate it.




\end{document}
 